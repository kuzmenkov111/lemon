% !TeX root = RJwrapper.tex
\title{\pkg{lemon} -- Freshing up your ggplots}
\author{by Stefan McKinnon Edwards}

\maketitle

\abstract{%
The ggplot2 package is highly popular for creat plots and offers many
options for customising the final product. There are however limitations
using only
  `core' ggplot2 functions,
  why users might have to modify the gtable object, which is returned when
  the plot is built for rendering.
  Where the ggplot2 plot describes how the data is visualised, the gtable
  contains the instructions for which lines, points, and labels to draw.
  This paper contains a brief introduction to the gtables that returned when
  the plot is built, while demonstrating some approaches for modifying the 
  gtable object.
  The paper also demonstrates some functions of the `lemon' package that are
  usefull for polishing the final product.
}

\section{Introduction}

The \CRANpkg{ggplot2} package \citep{ggplot2009,ggplot2016} is hugely
popular package for plotting in \R, and has seen a large number of
extensions. At the time of writing, \url{ggplot2-exts.org} features 29
packages, ranging from themes of popular sources (\CRANpkg{ggthemes},
\citealt{ggthemes}), to visualisation of graphs (\CRANpkg{ggraphs},
\citealt{ggraph}). Since version 2.0, it has featured an official
extension mechanism \citep{ggproto} that will see even more extensions
to \CRANpkg{ggplot2}.

\pkg{lemon} is just one of these packages with extensions.

The motivation for this package was to provide \emph{plot polishing}
functions. Polshing the final plot in preparation for submission can
often be acheived by subtly changing the plot's theme. Sometimes it is
however necessary to modify the drawn picture rather than the
description of the plot. When a \dfn{ggplot} plot is drawn, it is first
converted to a collection of graphical objects (\dfn{grobs}) contained
in a grob table (\dfn{gtable}). The grobs are known from the \pkg{grid}
package \citep{Murrell1999,Murrell2005,Murrell2011} and the gtables from
the \CRANpkg{gtable} package \citep{gtable}.

The \pkg{lemon} package supplies functions that allows us to polish the
final plot, either by modifying the gtable object
(e.g.~\code{reposition\_legend}, figure \ref{fig:reposition_legend}) or
by modifying how the gtable object is created when drawing axes
(e.g.~\code{capped\_horisontal} or \code{brackets\_vertical}, figure
\ref{fig:brackets}). This paper therefore gives a brief introduction to
\code{'grob'} and gtable objects.

After the introduction some of the \pkg{lemon} package's functionality
is demonstrated. A summary of the functions are supplied in table
\ref{tab:functions}.

\begin{table}
  \caption{Summary of functions supplied by \pkg{lemon}.}
  \label{tab:functions}

  \begin{tabular}{p{4.5cm}p{6.5cm}p{3cm}}  % page width is 14cm
    \toprule
      Function  & Description & Returns \\
    \midrule
      \code{g\_legend} & \rr Extracts the \samp{guide-box} component from a plot. & gtable \\
      \code{reposition\_legend} & \rr 
        Repositions the \samp{guide-box} onto e.g.\ the main graph. 
        Accepts both ggplot and gtable objects.             & gtable \\
      \code{grid\_arrange\_shared\_legend} & \rr 
        Like \code{gridExtra::grid.arrange}, but extracts legend from first argument
        and re-places it onto a margin.                                       & gtable \\
    \midrule
      \code{coord\_flex\_*}, \code{coord\_capped\_*} & \rr 
        Sets a Cartesian coordinate system for ggplot objects, 
        but accepts functions for drawing the axes.                           & Object inheriting from \code{'Coord'} class. \\
      \code{capped\_horisontal}, \code{capped\_horizontal}, \code{capped\_vertical} & \rr 
        Functions for drawing the axes. Can be supplied to the 
        \code{coord\_flex\_*} and \code{coord\_capped\_*} functions.        & Function \\
      \code{brackets\_horisontal}, \code{brackets\_horizontal}, \code{brackets\_vertical} & \rr 
        As above.                                                             & Function \\
    \midrule
      \code{facet\_rep\_grid}, \code{facet\_rep\_wrap} & \rr 
        As \CRANpkg{ggplot2}'s \code{facet\_grid} and \code{facet\_wrap}, but 
        draws \emph{all} axes and optionally tick labels.                     & Object inheriting from \code{'Facet'} class. \\
    \midrule
      \code{gtable\_show\_grill}, \code{gtable\_show\_names} & \rr
        As \CRANpkg{gtable}'s \code{gtable\_show\_layout} that visualises the
        structure.                                                            & gtable \\
    \bottomrule
  \end{tabular}
\end{table}

\section{A brief introduction to gtable}

We assume the reader is familiar with \CRANpkg{ggplot2}; else see e.g.
\citet{ggplot2016}. In short, it differs from R's native plotting
functions in that it uses a `grammar of graphics' to \emph{describe} the
visualisation we want, rather than instructions of which lines to draw
where.

This introduction gives a brief overview of the grobs and gtables.
\citet{Zhou2010} has a more general introduction to grobs, that also
includes \dfn{viewports}. Baptiste Augui\'{e}, author of
\CRANpkg{gridExtra} \citep{gridExtra}, has a guide on gtable objects at
\url{https://github.com/baptiste/gridextra/wiki/gtable}. We will however
focus this introduction on the gtable object that is returned from
ggplot.

The gtable has its introduction when a ggplot object is to be drawn on a
device. This causes ggplot2 to go through the motions of training the
object's scales on the data and doing its magic on translating the
grammar of graphics into instructions of which text, lines, points, and
polygons to draw where.

These instructions are described using grobs by the \pkg{grid} package.
At their simplests, they describe the coordinates where to draw a line
or a point together with the descriptions of line style, size, colours,
and font. The coordinates are usually relative to their frame of
reference (i.e.~the \dfn{viewport}), but can also be specified in
physical measures such as centimetres and inches (see
\code{?grid::unit}).

Additionally, a grob can itself be a collection of grobs. Not only does
this allow for organising the grobs in a hierarchical structure, but the
grobs' viewport may be defined relative to the grobs' parents' viewport.
The enables the user to create detailed drawings by specifying
composites that can be positioned relative to each other, and not just
relative to the entire canvas.

The gtable is also a collection of grobs, but with the added attributes
of arranging its collection of grobs (and grob collections) onto a
table-like layout of columns and rows. This eases the alignment of
different components, such as an axis with the plot.

To get us started, we create a simple scatter plot using
\CRANpkg{ggplot2} and then convert it to a gtable object:


\begin{Sinput}
library(ggplot2)
p <- ggplot(mtcars, aes(wt, mpg)) + geom_point() + labs(title='A scatter plot')
g <- ggplotGrob(p)
\end{Sinput}


The first object, \code{p}, is a ggplot object, and the second object,
\code{g}, a gtable object. Calling to print the objects, implicitly by
calling the object in an interactive session \{\R\} or directly with
\code{print}, we obtain the output given in figure \ref{fig:p_object}
for the ggplot object, and figure \ref{fig:g_object} for the gtable
object. To draw the gtable object, use \code{grid::grid.draw}.

\begin{figure}[h!]

\begin{Sinput}
print(p)
\end{Sinput}

\includegraphics{edwards2017-p_object-1} 
\caption{Printing the ggplot object.}
\label{fig:p_object}
\end{figure}

\begin{figure}[h!]

\begin{Sinput}
print(g)
\end{Sinput}
\begin{Soutput}
#> TableGrob (10 x 7) "layout": 17 grobs
#>     z         cells       name                                   grob
#> 1   0 ( 1-10, 1- 7) background        rect[plot.background..rect.864]
#> 2   5 ( 5- 5, 3- 3)     spacer                         zeroGrob[NULL]
#> 3   7 ( 6- 6, 3- 3)     axis-l    absoluteGrob[GRID.absoluteGrob.851]
#> 4   3 ( 7- 7, 3- 3)     spacer                         zeroGrob[NULL]
#> 5   6 ( 5- 5, 4- 4)     axis-t                         zeroGrob[NULL]
#> 6   1 ( 6- 6, 4- 4)      panel               gTree[panel-1.gTree.837]
#> 7   9 ( 7- 7, 4- 4)     axis-b    absoluteGrob[GRID.absoluteGrob.844]
#> 8   4 ( 5- 5, 5- 5)     spacer                         zeroGrob[NULL]
#> 9   8 ( 6- 6, 5- 5)     axis-r                         zeroGrob[NULL]
#> 10  2 ( 7- 7, 5- 5)     spacer                         zeroGrob[NULL]
#> 11 10 ( 4- 4, 4- 4)     xlab-t                         zeroGrob[NULL]
#> 12 11 ( 8- 8, 4- 4)     xlab-b titleGrob[axis.title.x..titleGrob.854]
#> 13 12 ( 6- 6, 2- 2)     ylab-l titleGrob[axis.title.y..titleGrob.857]
#> 14 13 ( 6- 6, 6- 6)     ylab-r                         zeroGrob[NULL]
#> 15 14 ( 3- 3, 4- 4)   subtitle  zeroGrob[plot.subtitle..zeroGrob.861]
#> 16 15 ( 2- 2, 4- 4)      title   titleGrob[plot.title..titleGrob.860]
#> 17 16 ( 9- 9, 4- 4)    caption   zeroGrob[plot.caption..zeroGrob.862]
\end{Soutput}

  \caption{Printing the gtable object.}
  \label{fig:g_object}
\end{figure}

The gtable object is of a completely different nature than the ggplot
object. The ggplot object contains the input data and descriptions of
the mappings between variables, geoms, and scales, and the gtable object
contains the instructions for drawing the lines, points, and text of the
plot. \strong{NB!} Once a ggplot object has been converted to a gtable
object, it cannot be converted back again, nor is it possible to add new
geoms or scales, or change the plot's theme. We can however add grobs or
modify the instructions of the gtable object.

\subsection{The gtable object}

A gtable is a hierarchical collection of grob and gtable objects that
arranges the plot's features onto a table-like layout of columns and
rows. The plot's features are the main graph, axes, legend, etc. The
most important components of a gtable object are the \samp{layout} data
frame and the \samp{grobs} list that positions and contains the plot's
features. These two are combined in the listing printed in figure
\ref{fig:g_object}. In addition to these, the \samp{widths} and
\samp{heights} unit lists describe the column widths and row heights.
This grid-structure is displayed figure \ref{fig:gtable_show_names},
superimposed on the plot.

\begin{figure}[h!]


\includegraphics{edwards2017-gtable_show_names-1} 
  \caption{
    Overlaying plot in figure \ref{fig:p_object} with the layout names and grill of the gtable object.
    Figure produced by \code{gtable\_show\_names} and \code{gtable\_show\_grill}.}
  \label{fig:gtable_show_names}
\end{figure}

The descriptors in column \samp{grob} are composites of each grob's
class and name. For \samp{background}, the grob is a primitive class
(\code{'rect'}). Other features are collections of grobs, such as the
panel and axes (\code{'absoluteGrob'} and \code{'titleGrob'},
respectively), and are both specialised classes of \pkg{grid}'s
\code{'gTree'}). Finally, the many rows with \code{'zeroGrob'} are
features that are not drawn. In other words, a \code{'zeroGrob'} is
simply a blank placeholder.

A feature can be drawn spanning multiple rows and/or columns.
\samp{background} shows this by spanning the entire plot in figure
\ref{fig:gtable_show_names}. In figure \ref{fig:g_object} we see this in
the row of the column \samp{cells} corresponding to \samp{background}
(i.e.~the top row). As multiple features can occupy the same cells
(e.g.~\samp{background} and \samp{panel}), the z-index (column \samp{z})
dictates the order in which they are stacked. Features with lowest
z-index are drawn first.

Although the order of the layout data frame and grobs list are
intrinsically linked, it is not enforced. The layout data frame can be
accessed via \code{g\$layout} and the grobs list via \code{g\$grobs},
and they can be manipulated separately as you normally would data frames
and lists, respectively.

\subsection{Widths and heights}

The gtable is also described by the heights and widths of rows and
columns. These two components are each a \code{'unit.list'} with a
component for each row or column. A quick glance at the widths of
\code{g} tells us a lot:


\begin{Sinput}
g$widths
\end{Sinput}
\begin{Soutput}
#> [1] 5.5pt                   1grobwidth              sum(1grobwidth, 2.75pt)
#> [4] 1null                   0cm                     0cm                    
#> [7] 5.5pt
\end{Soutput}


These can be compared to the columns of figure \ref{fig:gtable_show_names};
the first and last are the plot's margins, as given by \code{theme(plot.margin)}.
The second width, \code{1grobwidth}, refers to the actual width of the grob
containing the y-axis title 'mpg'.
The third width, \code{sum(1grobwidth, 2.75pt)}, 
is literally the width of the y-axis labels \emph{plus}
2.75\,pt, with this unit being the length of tick marks, as set by \code{theme(axis.tick.length)}.
We will in general see \samp{grobwidth} and \samp{grobheight} for text elements,
with the width and heigth giving the horisontal and vertical length of a box 
encompassing the text, taken into account if the text is rotated.
The fourth width, \code{1null}, simply expands into the available space and is here 
used to maximise the width of the actual graphing area.
The two remaining widths of 0 cm are those for columns for a right-hand side axis.
The columns are shown in figure \ref{fig:gtable_show_names} as they have been
expanded for us to see.

The number of columns and rows in the gtable object are not fixed to those shown here.
Adding e.g.\ a legend or using facetting will affect how many columns and rows are 
used in the gtable object. See example of figure \ref{fig:change_width} for 
assessing which row or column to use.


\subsection{Manipulating the gtable object}

The \CRANpkg{gtable} package supplies a number of functions for manipulating
a gtable, such as adding rows and columns (\code{gtable\_add\_rows},
\code{gtable\_add\_cols}), combining several gtables as if 
they were matrices (\code{cbind}, \code{rbind}),
or adding new features to the gtable object (\code{gtable\_add\_grob}).
These functions are usually sufficient for manipulating the gtable object.
Knowing the components of the gtable object however allows us to manipulate 
it in an alternative and direct fashion.

\subsubsection{Adding or replacing features}

For adding a feature in the form of a grob, \CRANpkg{gtable} comes with the function 
\code{gtable\_add\_grob}.
This appends a new row to the \samp{layout} data frame with the coordinates of the
spanning cells, a \samp{name} for the grob, 
and appends the added grob to the \samp{grobs} list.

\emph{Replacing a grob} is however more tricky. 
To do so, we have to work out which index in the \code{grobs} list corresponds 
to the feature that is to be replaced. 
The example in figure \ref{fig:replace_grob} demonstrates this for replacing the grob containing the 
x-axis labels, with  a pink box.

\begin{figure}[h!]

\begin{Sinput}
g <- ggplot2::ggplotGrob(p)
i <- which(g$layout$name == 'xlab-b')
g$grobs[[i]] <- grid::roundrectGrob(r=unit(0.5, 'snpc'), 
                                    gp=gpar(fill='pink', col='black'))
grid::grid.draw(g)
\end{Sinput}

\includegraphics{edwards2017-replace_grob-1} 

\caption{Replacing the x-axis label grob with a pink square.}
  \label{fig:replace_grob}

\end{figure}

One could naturally just have added the grob with
\code{gtable\_add\_grob}, placing it on top of the x-axis labels. But if
the grob was not filled with a solid colour, the labels could still
appear behind it. The example however also illustrates how to
\emph{extract} a grob, if this needed altering.

\subsubsection{Changing sizes}

\emph{Row heights} and \emph{column widths} can be directly altered by
changing the corresponding element of the \samp{heights} or
\samp{widths} \code{'unit.list'} lists. Although these are specified to
fit the plot's features while maximising the size of the panels, we can
change them.

To change the width of the panel we must change the width of the column
it resides in.

In the example of figure \ref{fig:change_width}, the width of the panel
is resized. To do so, the row in the \samp{layout} data frame
corresponding to the panel is located, which gives the row and column of
the top-left cell that the panel is drawn in. This is is in turn used
set a new width for the column.

\begin{figure}[h!] 
\begin{Sinput} 
g <- ggplot2::ggplotGrob(p)
layout <- subset(g$layout, name == 'panel')
g$widths[[layout$l]] <- unit(5, 'cm')
grid::grid.draw(g)
\end{Sinput}

\includegraphics{edwards2017-change_width-1}

\caption{The scatter plot with the width of the panel set to 5\,cm.}
  \label{fig:change_width} \end{figure}

As \code{g\$widths} is a \code{'unit.list'} object, it requires using
double square brackets for replacing an element. This also implies
changing the widths must be done one by one, or by building a new
\code{`unit.list'} object entirely.

\section{Demonstration}

The main topics for the \pkg{lemon} package was to provide functions for
working with the legend and axis lines, and are thus demonstrated in the
following. To have the axis lines repeated on all panels of facets, the
package provides functions that extends \CRANpkg{ggplot2}'s standard
facets, which is also demonstrated below.

\subsection{Legends}

The primary contribution here is the function \code{reposition\_legend}.
The default behaviour for \CRANpkg{ggplot2} is to place the legend in
one of the four margins of the plot. Repositioning the legend to occur
within the panel can be done by supplying a two-element numeric vector
to \code{theme(legend.position)}, specifying the position
\emph{relative to the entire plot}. This has the unfortunate effect that
the legend may not be positioned correctly after changing any size of
the plot (e.g.~resizing the plot's size or changing the font sizes).

The function \code{reposition\_legend} allows us to position the legend
in relative terms to any cell of the gtable. In order to do so, it
forces the plot into a gtable object, which is always returned
invisibly. The function is demonstrated in figure
\ref{fig:reposition_legend}.

\begin{figure}[hb!]

\begin{Sinput}
reposition_legend(p %+% aes(color=cyl), position = 'top right')
\end{Sinput}

\includegraphics{edwards2017-reposition_legend-1} 
  \caption{Repositioning the legend on to the main panel.}
  \label{fig:reposition_legend}
\end{figure}

The appearences of the legend is not modified in any way, thus modifying
it is done by the usual methods of \CRANpkg{ggplot2}. These could be
\code{theme(legend.direction)} or \code{guide\_legend(...)}.

We also highlight \code{theme(legend.box.background)} when having
multiple legends as the component \code{theme(legend.background)} only
affects the background of each singular legend, not the area spanning
all legends. Whilest on the topic of multiple legend, we also
demonstrate how \code{reposition\_legend} function can be used to add
multiple legends to different locations, as demonstrated in figure
\ref{fig:multiple_legends}.

\begin{figure}[h!]

\begin{Sinput}
x <- p %+% aes(color=cyl, size=qsec)
g <- g_legend(x)
x <- reposition_legend(x, position='bottom left', legend=g$grobs[[2]], plot=FALSE)
x <- reposition_legend(x, position='top right', legend=g$grobs[[1]], plot=FALSE)
grid.draw(x)
\end{Sinput}

\includegraphics{edwards2017-multiple_legends-1} 
  \caption{Repositioning multiple legends to different locations.}
  \label{fig:multiple_legends}
\end{figure}

The demonstration also shows the usage of \code{g\_legend} to extract
the legend from a plot. This functionwas first proposed by Baptiste
Augui\'{e} as early as June 2012 on \CRANpkg{ggplot2}'s Github
wiki\footnote{\url{https://github.com/tidyverse/ggplot2/wiki/Share-a-legend-between-two-ggplot2-graphs/047381b48b0f0ef51a174286a595817f01a0dfad}}.
It has since propogated throughout a swarm of Stack Overflow answers.
For this same reason, Baptiste Augui\'{e} has been added as contributor
to the \pkg{lemon} package.

Lastly, a modified version of \code{grid.arrange} with a shared legend
is demonstrated in figure \ref{fig:brackets}. The function
\code{grid\_arrange\_shared\_legend} supplied in this package is
modified from Shaun Jackman's
version\footnote{\url{http://rpubs.com/sjackman/grid\_arrange\_shared\_legend}}
and Baptiste Augui\'{e}
version\footnote{\url{https://github.com/tidyverse/ggplot2/wiki/Share-a-legend-between-two-ggplot2-graphs}},
thus Shaun Jackman has been added as a contributor too.

\subsection{Axis lines and brackets}

The second domain of functions in \pkg{lemon} is for axis lines and
brackets. We desired to alter the axis lines to be capped to the tick
marks or to be replaced by brackets.

In the following example, three functions are demonstrated: brackets,
capped axis lines, and multiple plots sharing the same legend.


\begin{Sinput}
p <- ggplot(mpg, aes(as.factor(cyl), hwy, colour=class)) +
  geom_point(position=position_jitter(width=0.3)) +
  theme_bw()

# Demonstrate brackets, left plot
p1 <- p + coord_flex_cart(bottom=brackets_horisontal(length=unit(0.08, 'npc'))) +
  theme(panel.border = element_blank(), axis.line = element_line(),
        panel.grid.major.x = element_blank())

p <- ggplot(mpg, aes(displ, hwy, colour=class)) + geom_point() +
  theme_bw()

# Demonstrate capped axis lines, right plot
p2 <- p + coord_capped_cart(bottom='right', left='none', gap = 0.03) +
  theme(panel.border = element_blank(), axis.line = element_line(),
        panel.grid.major.x = element_blank())

# Demonstrate shared legend
grid_arrange_shared_legend(p1, p2, position='bottom')
\end{Sinput}


\begin{figure}[h!]


\includegraphics{edwards2017-brackets-1} 
\caption{
  \emph{Left:} Displaying brackets on an axis to indicate categorical values.
  \emph{Right:} Capped axis lines, with x-axis capped at right-most tick, and
               y-axis capped by a small proportion.
  For displaying the axis lines, the theme's \samp{axis.lines} are enabled
  while \samp{panel.border} is disabled, as the latter is drawn on top of the former.
}
\label{fig:brackets}
\end{figure}

The \code{coord\_flex\_*} and \code{coord\_capped\_*} functions are
analogous to \CRANpkg{ggplot2}'s Cartesian Coordinate functions
(\code{coord\_cartesian}, \code{coord\_flip}, and \code{coord\_fixed}).
The \code{coord\_capped\_*} functions are wrappers for
\code{coord\_flex\_*}, which in turn returns classes inheriting from the
relevant Cartesian Coordinate classes (\code{'CoordCartesian'},
\code{'CoordFixed'}, and \code{'CoordFlip'}). The difference between the
\code{coord\_flex\_*} functions and their super classes is that these
accepts user defined functions for drawing the axes. The Coordinate
classes also have other responsibilities, e.g.~adjusting the position of
overlapping geoms, but with the \CRANpkg{ggplot2} v.2.0.0 extension
mechanisms, we can create subclasses and only alter the behaviour of the
\samp{render\_axis\_h} and \samp{render\_axis\_v} methods.

The custom axis drawing functions supplied to \code{coord\_flex\_*} and
\code{coord\_capped\_*} are called when the ggplot object is converted
to a gtable object for drawing. \CRANpkg{ggplot2} default function for
drawing axes is the non-exported \code{render\_axis}. Functions for our
customised axes can be created by calling e.g.~\code{capped\_horisontal}
or \code{brackets\_vertical}.

The appearences of the customised axis lines are taken from their
corresponding components of \code{theme}, so modifing their apparences
is consistent with \CRANpkg{ggplot2}. The bracket functions accepts
arguments to change the direction of the end
\texttt{ticks\textquotesingle{},\ as\ well\ as\ the\ length\ of\ the\ bracket\ and\ the\ length\ of\ the\ end}ticks'.

The use of axis lines to be more informative of the displayed data has
been shown in \citet[p.\,130]{Tufte2001} with \dfn{range-frames}, where
axis lines are limited to the range of the displayed data. This has been
implemented in \CRANpkg{ggthemes} \citep{ggthemes} with
\code{geom\_rangeframe}.

The capped axis lines differs from these by intending to display the
valid range the values may take. E.g.~correlations where the value
cannot exceed \(\pm 1\). The capped axis lines may also simply be used
to disconnect the axis lines at the corner, as seen in figure
\ref{fig:brackets}, right.

\subsection{Facets with repeated axis lines}

\CRANpkg{ggplot2}'s facets avoid displaying redundant axis lines and
labels, when the scale is identical along a row or column. This is
mostly a good idea, but as demonstrated, some effort has been put into
how the axis lines are drawn and as such, there is motivation to have
them displayed on all facet panels.

The function \code{facet\_rep\_grid} is demonstrated in figure
\ref{fig:facet_rep}. As with the \code{coord\_flex\_\*} functions, the
facets with repeated axis lines are analogous to \CRANpkg{ggplot2}'s
facet functions (\code{facet\_grid} and \code{facet\_wrap}), and are
implemented by subclassing \samp{FacetGrid} and \samp{FacetWrap}. They
accept the same arguments as their \CRANpkg{ggplot2} counterparts, in
addition to the argument \code{repeat.tick.labels}. The facets are
demonstrated in figure \ref{fig:facet_rep}.

\begin{figure}[h!]

\begin{Sinput}
df <- diamonds[sample.int(nrow(diamonds), 2000),]
d <- ggplot(df, aes(carat, price)) +
  geom_point(aes(colour = clarity)) + 
  coord_capped_cart(bottom='none', left='bottom') +
  facet_grid(.~cut) + theme_bw() + 
  theme(panel.border = element_blank(), axis.line = element_line(size = 0.4),
        legend.position='bottom')
d + facet_rep_grid(.~cut)
\end{Sinput}

\includegraphics{edwards2017-facet_rep-1} 
  \caption{Faceting with axis lines repeated across all panels.}
  \label{fig:facet_rep}
\end{figure}

\section{Availability}

The \pkg{lemon} package is currently available on CRAN.

\subsection{Acknowledgements}

This article was initiated with the \texttt{rticles} package
\citep{rticles}.

The author, SME, was funded by the MRC (FARSPhase: a Flexible, widely
Applicable, Robust, and Scalable phasing algorithm for human genetics,
grant no. MR/M000370/1), while working on this software.

\href{https://baptiste.github.io/}{Baptiste Augui\'{e}} for his work on
many functions for \CRANpkg{ggplot2} and \CRANpkg{grid} for which
without, this package could not have been possible.

\bibliography{edwards}

\address{%
Stefan McKinnon Edwards\\
The Roslin Insititute, University of Edinburgh\\
Easter Bush\\ Midlothian\\ EH25 9RG\\ Scotland, UK\\ true\\
}
\href{mailto:sme@iysik.com}{\nolinkurl{sme@iysik.com}}

